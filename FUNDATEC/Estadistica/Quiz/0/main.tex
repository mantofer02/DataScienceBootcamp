\documentclass{article}
\usepackage{graphicx} % Required for inserting images
\usepackage{babel}
\usepackage{geometry}
\usepackage{xcolor}
\usepackage{amsmath}
\usepackage{listings}
\usepackage{xcolor}

\definecolor{codegreen}{rgb}{0,0.6,0}
\definecolor{codegray}{rgb}{0.5,0.5,0.5}
\definecolor{codepurple}{rgb}{0.58,0,0.82}
\definecolor{backcolour}{rgb}{0.95,0.95,0.92}

\lstdefinestyle{mystyle}{
    backgroundcolor=\color{backcolour},   
    commentstyle=\color{codegreen},
    keywordstyle=\color{magenta},
    numberstyle=\tiny\color{codegray},
    stringstyle=\color{codepurple},
    basicstyle=\ttfamily\footnotesize,
    breakatwhitespace=false,         
    breaklines=true,                 
    captionpos=b,                    
    keepspaces=true,                 
    numbers=left,                    
    numbersep=5pt,                  
    showspaces=false,                
    showstringspaces=false,
    showtabs=false,                  
    tabsize=2
}

\lstset{style=mystyle}
\geometry{
    verbose,
    tmargin=3cm,
    bmargin=3cm,
    lmargin=2cm,
    rmargin=2cm,
    headheight=2cm,
    headsep=2cm,
    footskip=2cm
}

\title{Quiz 0}
\author{mantofer2000}
\date{April 2024}


\begin{document}

\begin{flushleft}
\begin{tabular}{|l|l|}
\hline 
 & \tabularnewline
\textbf{\large{}Instituto Tecnológico de Costa Rica} & QUIZ 0\tabularnewline
\textbf{\large{}Escuela de Computación} & Entrega: Domingo 14 de Abril, a través del TEC digital\tabularnewline
 & Debe subir un \emph{pdf }con la respuesta,\tabularnewline
Programa en Ciencias de Datos & generado con latex (adjunte los archivos .tex asociados).\tabularnewline
\textbf{Curso: Estadistica} & \tabularnewline
 & \tabularnewline
Profesor: Ph. D. Saúl Calderón Ramírez & Valor: 100 pts.\tabularnewline
 & Puntos Obtenidos: \_\_\_\_\_\_\_\_\_\_\_\_\tabularnewline
 & \tabularnewline
 & \tabularnewline
 & Nota: \_\_\_\_\_\_\_\_\_\_\_\_\_\_\_\_\tabularnewline
 & \tabularnewline
\cline{2-2} 
\multicolumn{2}{|c|}{}\tabularnewline
\multicolumn{2}{|l|}{Nombre del (la) estudiante: \textbf{Marco Ferraro Rodriguez}}\tabularnewline
\multicolumn{1}{|l}{} & \tabularnewline
\multicolumn{1}{|l}{Carné: \textbf{1 1782 1786}} & \tabularnewline
\multicolumn{1}{|l}{} & \tabularnewline
\hline 
\end{tabular}
\par\end{flushleft}
\begin{enumerate}
\item \textbf{(60 puntos)} Demuestre que el \emph{skew }o la inclinación
de una función de densidad exponencial: 
\[
p\left(x|\lambda\right)=\lambda e^{-\lambda x}
\]



\par es siempre $\gamma=2$, tomando en cuenta que $\mathbb{E}\left[X^{3}\right]=\frac{6}{\lambda^{3}}$. 

\par \textbf{Respuesta:} 

Con respecto a lo visto en clase sabemos que 
\[
\gamma = E\left[ \left( \frac{{X - \mu}}{\sigma} \right)^3 \right]
\]

\par Así mismo, también conocemos que para una función de distribución exponencial

\[
\mu = \frac{1}{\lambda}
\]

\par También, del material sabemos que la varianza de una función distribución exponencial es 

\[
Var[X] = \frac{1}{\lambda^2}
\]

\par Para hacer la comprobación, debemos tener algunas propiedades fundamentales, asi como de linealidad de la esperanza en mente 

\[
E[c] = c
\]
\[
E[cX] = cE[X]
\]
\[
E[X + Y] = E[X] + E[Y]
\]

\par Teniendo este en mente, y con el uso de propiedades, podemos expresar el skew de la siguiente forma:

\[
\gamma = E\left[ \left( \frac{{X - \mu}}{\sigma} \right)^3 \right]
\]

\[
\gamma = \frac{1}{\sigma^3}     E\left[ \left( X - \mu \right)^3 \right]
\]

\[
\gamma = \frac{1}{\sigma^3}     E\left[ X^3 - 3X^2\mu + 3X\mu^2 - \mu^3  \right]
\]

\[
\gamma = \frac{1}{\sigma^3}    ( E\left[ X^3] - 3E[X^2]\mu + 3E[X]\mu^2 - \mu^3  \right)
\]

\[
\gamma = \frac{1}{\sigma^3}    ( E\left[ X^3] - 3\mu(E[X^2] - E[X]\mu) - \mu^3  \right)
\]

\par Recordemos que para este caso, mu es equivalente a la esperanza de X. 

\[
\gamma = \frac{1}{\sigma^3}    ( E\left[ X^3] - 3\mu(E[X^2] - E[X]^2) - \mu^3  \right)
\]

\par Recordemos que la varianza puede expresarse en terminos de esperanza 

\[
Var(X) =  E[(X - E[X])^2]
\]


\[
Var(X) =  E[X^2 - 2XE(X) + E(X)^2]
\]

\[
Var(X) =  E(X^2) - 2E(X)E(X) + E(X)^2
\]


\[
Var(X) =  E(X^2) - E(X)^2
\]

\par Teniendo esto, podemos sustituir en nuestra ecuación

\[
\gamma = \frac{1}{\sigma^3}    ( E\left[ X^3] - 3\mu(Var(X) - \mu^3  \right)
\]


\[
\gamma = \frac{1}{\sigma^3}    ( E\left[ X^3] - 3\mu(\sigma^2) - \mu^3   \right)
\]

\par Remplazamos el supuesto que tenemos en el enunciado

\[
\gamma = \frac{1}{\sigma^3}    ( \frac{6}{\lambda^3} - 3\mu(\sigma^2) - \mu^3)
\]

\[
\gamma = \frac{1}{\sigma^3}    ( \frac{6}{\lambda^3} - \frac{3}{\lambda}(\sigma^2) - \frac{1}{\lambda^3})
\]

\par Recordemos que la desviación estándar es la raíz cuadrada de la varianza  

\[
\gamma = \frac{1}{\frac{1}{\lambda^3}  }    ( \frac{6}{\lambda^3} - \frac{3}{\lambda^3} - \frac{1}{\lambda^3})
\]


\[
\gamma = \lambda^3   ( \frac{2}{\lambda^3})
\]

\[
\gamma = 2
\]
\item \textbf{(40 puntos)} Con pytorch, genere 100 muestras de tamaño $N\text{=1000}$,
usando una densidad exponencial. Hagalo para dos valores diferentes
de $\lambda$ a su elección. Para esas muestras, calcule de forma
vectorial el sesgo $\gamma$, y verifique la demostracion anterior.
Adjunte el archivo jupyter con tal codigo. 
\end{enumerate}

\par \textbf{Respuesta (Archivo con Saluda Adjuntado):} 

\begin{lstlisting}[language=Python, caption=Creación de Muestras]
import torch
lambda_a = 0.5 
lambda_b = 173

torch.manual_seed(42)

num_samples = 100
N = 1000

samples_a = torch.empty(num_samples, N).exponential_(lambd=lambda_a)
samples_b = torch.empty(num_samples, N).exponential_(lambd=lambda_b)

print(samples_a.shape)
print(samples_b.shape)

\end{lstlisting}


\begin{lstlisting}[language=Python, caption=Función de Skewness]
def calculate_skewness(samples):
    mean = torch.mean(samples)
    std = torch.std(samples, unbiased=True)
    skewness = torch.mean(((samples - mean) / std) ** 3)
    return skewness
\end{lstlisting}

\begin{lstlisting}[language=Python, caption=Llamado de función]
skewness_a = calculate_skewness(samples_a.flatten())
skewness_b = calculate_skewness(samples_b.flatten())

print("Skewness for samples_a:", skewness_a)
print("Skewness for samples_b:", skewness_b)
\end{lstlisting}

\end{document}
